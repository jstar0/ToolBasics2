\thispagestyle{plain} % Page style without header and footer
\pdfbookmark[1]{概述}{概述} % Add entry to PDF
\chapter*{概述} % Chapter* to appear without numeration
\label{cp:abstract}

本实验报告使用由\textbf{于景一}制作的\LaTeX{}模板完成。关于此模板的信息,您可以前往\href{https://github.com/jstar0/LaTeXTemplate/}{GitHub模板仓库}具体了解。\footnote{或您可直接搜索GitHub账号\textit{@jstar0}了解更多}\footnote{您请注意,本模板基于LPPL v1.3c分发,本项目在原模板\href{https://github.com/joseareia/ipleiria-thesis}{Polytechnic University of Leiria: LaTeX Thesis Template}的基础上进行了合法地大量二改,包括但不限于自定义风格、中文化支持、样式重定义、功能增加等。}本文章使用的是“实验报告”模板。\footnote{模板提供两种样式,一种为学术论文样式,另一种为实验报告样式,具体区别请检查GitHub仓库上的两个分支。}\\

本实验报告是\textit{系统开发工具基础课程}的第二次实验报告,主要关于\textbf{掌握Shell的基本用法}\textit{,主要讨论各个发行版、数据管道、简单的Shell脚本、Shell内置的工具},学习使用Shell工具\textbf{进行数据整理},尝试并熟练\textbf{Vim的使用},总结实战经验,记录心得体会。\\

本实验报告的仓库地址为 \href{https://github.com/jstar0/ToolBasics2}{ToolBasics2 by jstar0}。\\

在\textbf{Shell的基本用法上},首先\textit{对各个发行版的Shell进行分析比对},主要包括\texttt{Bash, Zsh}等,并分析其\textit{工作原理与交互模式},了解主要的\textit{Shell内置工具},并\textit{学习数据管道的使用},主要是\texttt{|, >, >>, <}等,最后\textit{了解Shell脚本语法},并以\textit{系统运维为主题编写一个简单的Shell脚本}。

在\textbf{数据整理}方面,首先\textit{了解其基本概念},尔后,学习常用的\textit{数据处理工具},最后学习基本的\textit{正则表达式},并\textit{尝试与Shell管道组合使用}。\\

在\textbf{Vim的使用}方面,包括\textit{Vim的哲学及其类似TUI编辑器的介绍},\textit{了解并熟悉Vim的基本操作},完成\texttt{vimtutor}.\\

\note{知识之海是无边无际的,只是尽力游弋,就已倍感费时费力,然而学习的过程是美好的。本次实验在Shell和Vim方面的粗浅研究,仅可窥得其冰山一角。本文十分惭愧地呈现了我在实验中对两个领域的浅薄了解,如有谬误还请批评斧正。}